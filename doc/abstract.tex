\begin{center}
  \textsf{\LARGE Perceiving and memorising packaged goods \\ in a supermarket environment}\\[1cm]
  \textsc{University of Bremen}\\
  \textsc{Diploma Thesis}\\[1cm]
  08.10.2013\\
  Carsten K\"onemann\\[3cm]
\end{center}

\vfill

\section*{Abstract}
\begin{multicols}{2}
  In the field of robotics, one of the most frequently performed tasks is the search for specific objects. It is an integral part of almost everything a robot does, because it first needs to know where things are in order to work with them. Current solutions either require the robot to search for objects as needed or utilize a lookup-table of objects and their locations. The former solution is very flexible but time-consuming, especially in larger environments. The latter is extremely fast, but also inflexible and high-maintenance. In this work, a new method is developed, that combines the advantages of both types of existing solutions: A system that continuesly looks out for objects that might become interesting in the future and dynamically creates a semantic map of its environment. Then, if a robot needs something it came across earlier, it already knows where to go.
\end{multicols}

\null\cleardoublepage
\null\clearpage


\begin{center}
  \textsf{\LARGE Perceiving and memorising packaged goods \\ in a supermarket environment}\\[1cm]
  \textsc{University of Bremen}\\
  \textsc{Diploma Thesis}\\[1cm]
  08.10.2013\\
  Carsten K\"onemann\\[3cm]
\end{center}

\vfill

\section*{Zusammenfassung}
\begin{multicols}{2}
  Eine der von Robotern am h\"aufigsten durchgef\"uhrten T\"atigkeiten ist die Suche nach bestimmten Objekten. Sie ist ein wesentlicher Bestandteil von fast allen anderen Aufgaben eines Roboters, da dieser zun\"achst einmal wissen muss wo sich Dinge befinden, bevor mit ihnen arbeiten kann. Bestehende L\"osungen verlangen entweder vom Roboter Objekte jedesmal zu suchen, wenn sie gebraucht werden, oder sie verwenden Nachschlagetabellen, in denen zu jedem Objekt dessen Position zu finden ist. Erstere L\"osung ist sehr flexibel, aber auch sehr zeitaufw\"andig, besonders in gr\"o\ss eren Umgebungen. Letztere ist extrem schnell, aber auch sehr unflexibel und wartungsintensiv. In dieser Arbeit wird eine neue Methode entwickelt, die die Vorteile der beiden Arten existierender L\"osungen vereint: Ein System das laufend Ausschau h\"alt nach Objekten in Zukunft gebraucht werden k\"onnten und dynamisch eine semantische Karte der Umgebung erstellt. Wenn ein Roboter dann etwas ben\"otigt, das er zuvor schon einmal gesehen hat, dann wei\ss\ er bereits wo es zu finden ist.
\end{multicols}

\null\cleardoublepage
\null\clearpage
